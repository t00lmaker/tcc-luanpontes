% ----------------------------------------------------------
% Introdução
% ----------------------------------------------------------
\chapter{Introdução}

Após a Segunda Guerra Mundial, os processos de urbanização e industrialização do Brasil, levaram milhões de pessoas do campo à cidade. 
Segundo o IBGE, nos anos 60, mais da metade da população brasileira era considerada rural, pouco mais de 50 anos depois, 84,4\% da população 
brasileira reside em áreas urbanas \cite{golan2015crowdfunding}. Os investimentos em infraestruturas não acompanharam o ritmo acelerado dessa mudança,
o que levou ao agravamento dos problemas urbanos, incluindo os relacionados à mobilidade e mais diretamente o transporte público. 

Diversos problemas afetam os passageiros em seu dia a dia, no entanto a superlotação provavelmente é o que mais interfere na qualidade de vida dos usuários. 
Algumas empresas de ônibus, a fim de solucionar tal problema, colocam diversos veículos na mesma linha em um curto intervalo de tempo durante os horários de pico, 
mas a ansiedade do usuários de chegar em casa, a instabilidade do horário dos ônibus e a incerteza se o ônibus extra já passou ou não, faz com que os passageiros normalmente 
peguem o primeiro veículo disponível, causando o mau aproveitamento da frota. Nessa caso a falta de informação sobre os veiculos disponíveis tem impacto 
direto sobre qualidade de vida dos usuário.


A espera de ônibus faz os usuário se exporem muitas vezes a uma situação de risco e desconforto. Cada minuto em um ponto de ônibus o passageiro fica exposto a 
intempéries do tempo, como chuva e sol e não raro, assaltos são registrados em tais locais. Muitas vezes o usuário obriga-se a estar na parada muito tempo antes do ônibus passar, 
por vezes, deixando de fazer suas atividades, perdendo sono ou deixando de passar mais tempo com a família, caracterizando outro impacto negativo que falta de informação sobre o
transporte público pode ter sobre a qualidade de vida das pessoas. Aumentar a frota de veículos para diminuir o tempo de espera pode ser economicamente inviável, 
mas oferecer a informação da quantidade de veículo disponíveis, bem como suas respectivas localizações e horários, pode ajudar a diminuir a quantidade de tempo que o usuário precisará 
ficar na parada sem elevar consideravelmente os custos operacionais, possibilitando até mesmo a diminuição dos preços  para o usuário final por tonar possível a diminuição da frota. 

Do ponto de vista das autoridades que cuidam do sistema como um todo, existe a dificuldade de levantar problemas em pontos específicos do sistema. Por exemplo, como o órgão responsável 
pelos os pontos de ônibus da cidade pode saber se os tais atendem as expectativas dos usuários? Ou como saber a visão do usuário a respeito da frota de veículos de uma determinada 
linha ou empresa? Atualmente cada órgão possui seu canal de contato com o usuário, que poderiam colher tais informações, no entanto tais canais requerem um esforço considerável do usuário, 
que pode desestimular o engajamento e a perda do sentimento que levaria ele a apontar problemas. Por outro lado as atuais plataformas digitais permitem o aproveitamento do sentimento de 
engajamento de forma imediata e com um esforço cosideravelmente menor. 

O presente trabalho descreve o desenvolvimento de uma plataforma que ofereça informações sobre o transporte público a desenvolvedores de aplicativos móveis, além oferecer um repositório 
centralizado de avaliações realizadas por usuário sobre diversos aspectos do transporte público. Com tal ferramenta os usuários poderão ter acesso fácil a informações sobre os
horários, intinerarios, localizão de veiculos e linhas por paradas,  que podem ajudar a resvolver os problemas citados anteriormente. Por outro lado, os mesmos poderão conceber uma 
visão sobre sistema transporte público local, por avaliarem pontos de ônibus ou estado de conservação dos veículos, informações essas que podem ser uteis as entidades que cuidam do 
transporte público da cidade.  

\section{Objetivos}

Abaixo são listados os objetivos gerais e específicos do presente trabalho.

\subsection{Objetivo Geral}

O objectivo geral deste trabalho é descrever o desenvolvimento de uma plataforma que centralize os dados do transporte público de Teresina e que facilite o desenvolvimento de novas aplicações 
voltadas para a melhoria da mobilidade urbana. 

\subsection{Objetivos Específicos}
\begin{lista}
  \item Descrever os processo de desenvolvimento de uma REST API web, incluindo as decisões tecnicas tomadas para a mesma. 
  \item Demostrar o funcionamento da plataforma criada.
  \item Expor a documentação da API exposta. 
\end{lista}