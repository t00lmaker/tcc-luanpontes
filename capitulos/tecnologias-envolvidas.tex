% ----------------------------------------------------------
% Tecnologias Envolvidas
% ----------------------------------------------------------
\chapter{Tecnologias Envolvidas} \label{cha:tecnologias-envolvidas}

As tecnologias utilizadas na construação da plataforma Starbus são divididos em três grupos de ferramentas. 
Abaixo é específicado cada grupo, bem como as ferramentas pertencentes a cada um deles. 

O primeiro grupo abrange as ferramentas utilizadas para a construação da plataforma em si, sendo a linguagem
de programação, o framework e banco de dados utilizado na aplicação. A liguagem Ruby na versao 2.2 é uma linguagem 
de curta curva de aprendizagem, dispondo de um conjunto de ferramentas maduras e com abordagem pragmática para 
desenvolvimento. Faz parte de seu ferramental o framework Grape, que tem se mostrado uma solução prática para 
a construção de uma API REST, focando em oferecer uma interface simples para manipulação de dados oriundos de 
requições web. Por isso, Grape foi utilizado na concepção da API REST da plataforma. O dados são armazenados em 
bando de dados Postgres, na versão 12.1, que é de fácil configuração e possui uma comunidade ativa. 

O segundo grupo é composto por ferramentas de suporte ao desenvolvimento. A primeira escolha foi o editor Vim que 
possui uma excelente suporte ao Ruby, além de permitir uma fácil alternacia entre o terminal e o editor, tornando 
pratica a execução de tarefas no terminal, que é tão comum no desenvolvimento Ruby. Para escrita dos testes da 
aplicação foi escolida a biblioteca Rspec na versao 3.9, que tem influenciado toda a comunidade de desenvolvimento
de software com sua expressividade e clareza. Para acesso, debug e manipulação de dados no Postgres, foi utilizada 
a IDE DBeaver, na versao 7.0, um software open source com funcionalidades excelentes.     

Por fim, para executar a plataforma Starbus em ambiente produtivo e assim atingir os objetivos específicos do 
trabalho, procurou-se uma ferramenta que pudesse trazer a menor carga de trabalho relacionada a infra estrutura 
de aplicação, visto que o foco do trabalho deveria ser no desenvolvimento da solução. Assim sendo, a plataforma 
Heroku foi a solução mais viável, visto que oferece um ambiente extremamente flexível e escalável com apenas poucas 
linha de configuração. Por exemplo, para se colocar no ar uma nova versão da aplicação basta realizar um git push para 
um determinado repositório da ferramenta.