%% Resumo
\begin{resumo}

% Falar do bitcoin e suas possibilidades, caracteristicas do teleram que justificam o uso do mesmo, chatbot, , da possibilidade de consumir APIs e Websockets de exchanges, assim como automatizar serviços até então manuais.

A internet e o e-mail revolucionaram a comunicação. Antes para enviar uma mensagem a uma pessoa de outro lado da Terra era necessário fazer isso pelos correios. Você dependia de um intermediário para fisicamente entregar uma mensagem. Retornar a esta realidade é inimaginável. Com as Criptomoedas você pode transferir fundos de A para B em qualquer parte do mundo sem jamais precisar confiar em um terceiro para esta simples tarefa. 

Resolvendo problemas que antes inviabilizava a utilização de moedas digitais sem a necessidade de um intermediário, como o gasto duplo e a transparência.  O Bitcoin deu abertura a um leque de possibilidades impulsionado pela facilidade no desenvolvimento de novas tecnologias. Surgindo assim inúmeros serviços, moedas e aplicações.

Uma prática já utilizada em corretoras do mundo todo é a criação de salas e grupos para discussão e análise dos preços dos ativos. Prática esta que veio a acontecer neste novo mercado impulsionada pela facilidade de comunicação através de aplicativos de mensagens instantâneas.

Visto este cenário, este trabalho apresenta o desenvolvimento de uma ferramenta de chatbot, que tem como principais funcionalidades a gerência de clientes e a criação, replicação e automação de operações de compra e venda de criptomoedas. Destinada a administradores de grupos, na plataforma de mensagens instantâneas Telegram, que de modo geral divulgam mensagens indicando quais moedas comprar e qual o melhor horário para vender as mesmas, buscando assim que seus assinantes, realizem lucros com operações de curto prazo aproveitando-se da volatilidade do mercado.

\vspace{\onelineskip}
\noindent
\textbf{Palavras-chaves}: Chatbot, Criptomoeda, Automação, Trader, Telegram.
\end{resumo}